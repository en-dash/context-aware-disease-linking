\PassOptionsToPackage{utf8}{inputenc}
\documentclass{bioinfo}
\copyrightyear{2015} \pubyear{2015}

\access{Advance Access Publication Date: Day Month Year}
\appnotes{Manuscript Category}


% Notes

% Citation styles:
% \cite:    Foo et al. (2018)
% \citep:   (Foo et al., 2018)
% \citealp: Foo et al., 2018

% Equations:
% use \begin{equation}

% Tables:
% \begin{table}[!t]
% \processtable{This is table caption\label{Tab:01}}
% {\begin{tabular}{@{}llll@{}}\toprule head1 &
% head2 & head3 & head4\\\midrule
% row1 & row1 & row1 & row1\\
% row2 & row2 & row2 & row2\\
% row3 & row3 & row3 & row3\\
% row4 & row4 & row4 & row4\\\botrule
% \end{tabular}}{This is a footnote}
% \end{table}

% Figures:
% \begin{figure}[!tpb]%figure2
% \centerline{\includegraphics{fig02.eps}}
% \caption{Caption, caption.}\label{fig:02}
% \end{figure}

% Sectioning:
% use \section, \subsection, \subsubsection (numbered non-star version)

% The submission page (https://academic.oup.com/bioinformatics/pages/instructions_for_authors) says we should use:
% Introduction, System and methods, Algorithm, Implementation, Discussion, References
% The template has:
% Introduction, Approach, Methods, Discussion, Conclusion
% But recently published articles show this:
% Introduction, Materials and methods, Results, Discussion (https://doi.org/10.1093/bioinformatics/bty179, .../bty193, .../bty117)


% The template wraps parts of the Methods section in \begin{methods} --
% why? why only parts of it?

% Template uses \enlargethispage{12pt}
% Template uses \vadjust{\newpage} and \vadjust{\pagebreak} (what do they do?)


\newcommand{\url}[1]{\href{#1}{#1}}
\newcommand{\eg}{e.\,g.\ }
\newcommand{\ie}{i.\,e.\ }


\begin{document}
\firstpage{1}

\subtitle{Data and text mining}

\title[Context-aware disease linking]{Context-aware disease linking}
\author[Sample \textit{et~al}.]{Corresponding Author\,$^{\text{\sfb 1,}*}$, Co-Author\,$^{\text{\sfb 2}}$ and Co-Author\,$^{\text{\sfb 2,}*}$}
\address{$^{\text{\sf 1}}$Department, Institution, City, Post Code, Country and \\
$^{\text{\sf 2}}$Department, Institution, City, Post Code,
Country.}

\corresp{$^\ast$To whom correspondence should be addressed.}

\history{Received on XXXXX; revised on XXXXX; accepted on XXXXX}

\editor{Associate Editor: XXXXXXX}


\abstract{%
\textbf{Motivation:} This section should specifically state the scientific question within the context of the field of study.\\
\textbf{Results:} This section should summarize the scientific advance or novel results of the study, and its impact on computational biology.\\
\textbf{Availability and Implementation:} This section should state software availability if the paper focuses mainly on software development or on the implementation of an algorithm. Examples are: 'Freely available on the web at \url{http://www.example.org}'; 'Website implemented in Perl, MySQL and Apache, with all major browsers supported'; or 'Source code and binaries freely available for download at URL, implemented in C++ and supported on linux and MS Windows'. The complete address (URL) should be given. If the manuscript describes new software tools or the implementation of novel algorithms the software must be freely available to non-commercial users. Authors must also ensure that the software is available for a full TWO YEARS following publication. The editors of Bioinformatics encourage authors to make their source code available and, if possible, to provide access through an open source license (see www.opensource.org for examples).\\
\textbf{Contact:} Full email address to be given, preferably an institution email address.\\
\textbf{Supplementary information:} Links to additional figures/data available on a web site, pr reference to online-only supplementary data available at the journal's web site.
}

\maketitle

\section{Introduction}

introduce task (linking vs. NER)

discuss state of the art (nobody uses context!)



\section{Materials and methods}

\subsection{Datasets}

We evaluated the proposed system using two annotated document collections.
The ShARe corpus consists of clinical reports with annotations for diseases and disorders, which were created for the ShARe/CLEF eHealth Evaluation Lab 2013 shared task \citep{pradhan-et-al:2013:CLEF}.
The NCBI disease corpus \citep{islamaj-dogan-et-al:2014} provides disease annotations over abstracts of scientific articles.
Both collections include references to the relevant mentions in their original context and link them to identifiers of a standard knowledge base.
In both cases, a predefined division into training and test set allows comparison across different approaches.
More detailed statistics on the datasets are given in Table~\ref{tab:datasets}.

The ShARe corpus contains de-identified clinical reports from US intensive care.
It is distributed as part of the MIMIC database  % TODO: reference
and is available for research purposes after acquiring a personal license.%
\footnote{Instructions for obtaining the data are given on the shared-task website, currently hosted at \url{https://sites.google.com/site/shareclefehealth/data}.}
The annotated mentions are grounded in the SNOMED CT terminology  % TODO: reference
using UMLS  % TODO: reference
identifiers (CUI).
Concepts that were not represented in SNOMED at the time of annotation were given the NIL symbol “CUI-less”.

The NCBI disease corpus comprises scientific abstracts and annotations which are freely available online.
In addition to the training and test set, a predetermined development set is provided (regarded as part of the training set in Table~\ref{tab:datasets}).
The mentions are linked to the MEDIC vocabulary  % TODO: reference
using MeSH  % TODO: reference
and OMIM  % TODO: reference
identifiers.
Unlike the ShARe corpus, the NCBI annotators were required to always provide an identifier, \ie there is no NIL label.
Furthermore, they were asked to map a mention to multiple identifiers if there was no single terminology entry that sufficiently covered the meaning of the concept in question.
As another difference, composite mentions such as “breast and ovarian cancer” are represented as a single annotation with a series of identifiers, whereas the ShARe corpus uses multiple, partially overlapping annotations (“breast [\dots] cancer” and “ovarian cancer”).

\begin{table}[!t]
\processtable{Dataset statistics.\label{tab:datasets}}
{\begin{tabular}{@{}llll@{}}\toprule
head1 & head2 & head3 & head4\\\midrule
row1 & row1 & row1 & row1\\
row2 & row2 & row2 & row2\\
row3 & row3 & row3 & row3\\
row4 & row4 & row4 & row4\\\botrule
\end{tabular}}{}
\end{table}


\subsection{System architecture}

We approach the entity linking task in a three-stage process:
generating candidate names (Section~\ref{ssub:cand-gen}),
ranking candidates (\ref{ssub:ranking}), and
mapping names to identifiers (\ref{ssub:id-mapping}).
In the candidate-generation stage, different strategies are used to extract a selection of synonyms from the knowledge base for each mention.
For the second stage, a CNN is trained to assign a score to each proposed mention-synonym pair.
In the last stage, the scored list is examined to pick the most likely identifier for each mention.

\subsubsection{Candidate generation}
\label{ssub:cand-gen}

The candidate generation phase is a preprocessing step for the subsequent ranking task.
It aims at keeping the ranking lists at a manageable size without missing relevant names.
A good recall value is crucial for the candidate generation, as it poses a hard upper boundary on the entire pipeline.

We implemented a candidate generator based on orthographic similarity, another one on semantic similarity, and a number of ancillary generators targeting specific cases.
For the orthographic similarity, mentions and candidates are represented as vectors of their character skip-grams (\ie n-grams with gaps),  % TODO: reference
which allows efficient computation of pair-wise cosine similarity.
The shape of the skip-grams, a similarity threshold as well as a cut-off value are experimental parameters that control the number of candidates returned for each mention.
For the semantic similarity, mentions and candidates are represented as phrase vectors which are obtained by averaging over the word vectors of pretrained embeddings.
The most similar candidates are determined through pair-wise cosine, analogously to the orthographic similarity.
The ancillary generators are concerned with specific tasks like abbreviation expansion, digit/numeral replacement, hyperonym relation, and composite mentions.
They operate in their respective niche and are triggered only in certain circumstances.

The generators provide a score for each candidate.
The cosine-based generators return a value in the range $[0,1]$, while others report $1$ for every candidate they can find.
If multiple generators produced the same candidate, their scores are aggregated; a $0$ score is substituted for each generator that did not produce the candidate in question.
For each candidate, a vector of scores is provided to the CNN, along with the candidate name.

oracle mode!

\subsubsection{Ranking}
\label{ssub:ranking}

based on \cite{lihaodi-et-al:2017}, but without the rule-based system --
we use candidate generators instead (3 types) --
in addition to Li et al., we use context (whole abstract/report and dict definitions)

Detailed description:
network layers (graphic!) --
candidate generators (surface sim., semantic sim., rule-inspired) --
context encoding (simple)

\subsubsection{Identifier mapping}
\label{ssub:id-mapping}




\section{Results}

ShARe/CLEF: better than Li et al. and D'Souza \& Ng.

NCBI: better than Li et al., but we didn't beat TaggerOne
(where do these 88.8\% come from? -- no answer from Li et al.)



\section{Discussion}

explain and regret need for candidate generators

maybe analyse contribution of individual candidate generators (like feature ablation)?



\section*{Acknowledgements}

Submission guidelines:
Please ensure you acknowledge all sources of funding.
Details of all funding sources for the work in question should be given in a separate section entitled 'Funding'. This should appear before the 'Acknowledgements' section.

But: the template and published examples list funding later.
Nobody acknowledges source of funding.
\vspace*{-12pt}



\section*{Funding}

This work was supported by \dots

An example is given here: ‘This work was supported by the National Institutes of Health [AA123456 to C.S., BB765432 to M.H.]; and the Alcohol \& Education Research Council [hfygr667789].’
\vspace*{-12pt}



\bibliographystyle{natbib}
\bibliography{refs}


\end{document}
